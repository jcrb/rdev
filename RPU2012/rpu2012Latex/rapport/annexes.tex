
\chapter{R??sum?? de passage aux urgences (RPU)}
La composition d'un RPU r??pond ?? une norme d??finie par l'INVS\footnote{Institut National de Veille Sanitaire} dont la derni??re version est dat??e de 2006. Un RPU se compose des ??l??ments suivants:
\begin{enumerate}
  \item premier
\end{enumerate}

\chapter{Documentation interne}
\begin{enumerate}
  \item Eurostat: Resural\//Stat Resural\//Eurostat\//eurostat\_readme.Rmd
  \item INSEE
  \item Open Street Map (OSM)
  \item cran-R
\end{enumerate}

\section{Logiciel R}
R est un langage de programmation et un environnement math??matique utilis??s pour le traitement de donn??es et l'analyse statistique. C'est un projet GNU fond?? sur le langage S et sur l'environnement d??velopp?? dans les laboratoires Bell par John Chambers et ses coll??gues. 
\
R est un logiciel libre distribu?? selon les termes de la licence GNU GPL et est disponible sous GNU/Linux, FreeBSD, NetBSD, OpenBSD, Mac OS X et Windows. R s'interface directement avec la pluspart des bases de donn??es courantes: BO (Oracle), MySQL, PostgreeSql, etc. Il s'interface aussi avec un certain nombre de syst??me d'information g??ographique (SIG) et sait lire nativement le format Shapefile utilis?? par l'IGN.
\
Le logiciel R est interfac?? avec le traitement de texte Latex par l'interm??diaire de la biblioth??que Sweave. Cette association permet de m??langer du texte et des formules math??matiques produisant les r??sultats et graphiques de ce document. En cas de modification des donn??es, il suffit de recompiler le fichier source pour mettre ?? jour le document final.
\
