\documentclass[12pt,english,french]{report}
\usepackage[francais]{babel}
\usepackage[T1]{fontenc}
\usepackage{lmodern}
\usepackage[utf8]{inputenc}
\include{rpu-2012.sty}
\usepackage{numprint}
\usepackage{makeidx}
\makeindex
\makeglossary

%
% Début du document
% ------------------
%
\usepackage{Sweave}
\begin{document} 
\Sconcordance{concordance:rpu2012.tex:rpu2012.Rnw:%
1 18 1 1 0 12 1 1 12 1 5 69 1 1 4 1 1 1 9 20 0 1 2 6 1 1 8 31 0 1 2 10 %
1 1 4 10 1 1 6 26 1 1 4 7 1 1 8 14 0 1 2 7 1 1 7 14 1 2 2 13 1 1 2 19 0 %
1 1 1 2 3 1}
\Sconcordance{concordance:rpu2012.tex:./rpu2012Latex/rapport/annexes.Rnw:ofs 300:%
1 21 1}
\Sconcordance{concordance:rpu2012.tex:rpu2012.Rnw:ofs 322:%
281 3 1}


\title{Analyse des données RPU 2012 de la région Alsace}
\author{RESURAL}
\date{\today}
\maketitle

\tableofcontents
\listoftables
\listoffigures



%\frontmatter

%\mainmatter

\section{Préambule}

Le présent document est une analyse partielle et incomplète des résumés des passages aux urgences (RPU) produits par une partie des hôpitaux d'Alsace ayant une autorisation de faire fonctionner un service d'urgence au cours de l'année 2012. C'est la première année que cette ébauche d'analyse est faite. En raison du caractère très incomplet des données, il ne peut être
tirés de conclusions définives sur le fonctionnement des urgences. C'est un jeu d'essai qui permet d'apprécier comment les RPU sont compris et saisis, quelles sont les difficultés potentielles sur lesquelles doit porter un effort pédagogique et final permettre d'ébaucher une stratégie d'analyse de ces informations.

\chapter{La région Alsace}
\section{Les secteurs sanitaires}
\section{Les zones de proximité}
\section{Démographie}
Les calculs sont effectués à partir du fichier xxx de l'INSEE qui recense l'ensemble de la population par commune et par tranches de un an. La version utilisée est celle du 1er janvier 2010 (tab.\ref{pop}).

\begin{table}
\begin{center}
\begin{tabular}{|l|l|r|r|}
  \hline
  Tranche d'age & Abréviation & Effectif & Pourcentage \\
  \hline
  \hline
  Moins de 1 an & pop0 & $21903.14$ & 1.19 \\
  De 1 à 75 ans & pop1\_75 & $1690073.00$ & 92.00 \\
  Plus de 75 ans& pop75 & $125110.90$ & 6.81 \\
  \hline
  Total & pop\_tot & $1837087.00$ & 100.00 \\
  \hline
\end{tabular}
\caption{Population d'Alsace (janvier 2010)}
\label{pop}
\end{center}
\end{table}

\section{Les services d'accueil des urgences (SAU)}

\begin{table}
\begin{center}
\begin{tabular}{|c|c|c|c|l|}
  \hline
& Finess utilisé & Finess géographique & Finess Juridique & Structure \\
  \hline
  \hline
1 & 670780055 &   & 670780055 & HUS \\
2 & 670780543 & 670000272 & 670780543 & CH Wissembourg \\
3 & 670000397 & 670000397  & 670780691 & CH Selestat \\
4 & 670780337 & 670000157 & 670780337 & CH Haguenau \\
5 &   & 670000165 & 670780345 & CH Saverne \\
6 & 670016237  & 670016237  & 670016211 & Clinique ste Odile \\
7 &   & 670780212 & 670014604 & Clinique Ste Anne \\
8 & 680000973 & 680000684 & 680000973 & CH Colmar \\
9 & 680000197  & 680000197  & 680000049 & Clinique des trois frontières \\
10 & 680000486 & 680000544  & 680000395 & CH Altkirch \\
11 & 680000700 & 680000700 & 680001005 & CH Guebwiller \\
12 & 680000627 & 680000627 & 680000486 & CH Mulhouse FG \\
13 &   & 680000601 & 680000437 & CH Thann \\
14 &   & 680000320  & 680000643 & Diaconat-Fonderie (St Sauveur) \\
\hline
\end{tabular}
\caption{Service d'accueil des urgences d'Alsace}
\label{summary}
\end{center}
\end{table}

%
% chapitre: les données
% ---------------------
%
\chapter{Les données}

\section{Origine des données}


Les données proviennent des RPU produits par les hôpitaux d'Alsace ayant l'autorisation de faire fonctionner un service d'urgence (SU). La liste des structures hospitalières ayant fournit des informations alimentant le présent rapport est fournie par la table \ref{tab1}, page \pageref{tab1}.


